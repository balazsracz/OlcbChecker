\documentclass[11pt]{article}
\usepackage{geometry}                % See geometry.pdf to learn the layout options. There are lots.
\geometry{letterpaper}                   % ... or a4paper or a5paper or ... 
%\geometry{landscape}                % Activate for for rotated page geometry
\usepackage[parfill]{parskip}    % Activate to begin paragraphs with an empty line rather than an indent
\usepackage{graphicx}
\usepackage{amssymb}
\usepackage{epstopdf}
\usepackage{hyperref}
\DeclareGraphicsRule{.tif}{png}{.png}{`convert #1 `dirname #1`/`basename #1 .tif`.png}

\usepackage{fancyhdr}
\fancyhf{}
\fancyfoot[R]{\thepage}
\renewcommand\headrulewidth{0pt}
\pagestyle{fancyplain}
% To put the name in the footer, do 
% \fancyfoot[L]{Foo}

% common header content elements
%\author{The OpenLCB Group}
%\date{}                                         % Activate to display a given date or no date



% to format XML tags in proper angle brackets <>
\newcommand*{\xml}[1]{\texttt{<#1>}}

% macro for inserting common introduction text
% takes hyperlink to a stadnard and the standard name as arguments
\newcommand{\introductionCaveats}[2]{

    \section{Introduction}

    This note documents the procedure for checking an OpenLCB implementation against the 
    \href{#1}
    {#2}.

    The checks are traceable to specific sections of the Standard.

    The checking assumes that the Device Being Checked (DBC) is being exercised by other
    nodes on the message network, 
    e.g. is responding to enquiries from other parts of the message network.
}

% macro for inserting the procedure section
\newcommand{\checkProcedure}[1]{
%    Select ``#1" in the program, 
%    then select each section below in turn.  Follow the prompts
%    for when to reset/restart the node and when to check 
%    outputs against the node documentation.
}

% macro deciding whether the pipset footnote should appear or not
\newcommand{\pipsetFootnote}{
%\footnote{Using the -p option or setting the checkpip default value False will skip this check.}
}

\title{Checking the OpenLCB Firmware Upgrade Protocol Standard}

\begin{document}
\maketitle


\introductionCaveats
    {https://nbviewer.org/github/openlcb/documents/blob/master/drafts/FirmwareUpgradeS.pdf}
    {(Draft) Firmware Upgrade Protocol Standard}

\section{Firmware Upgrade Protocol Procedure}

\checkProcedure{Firmware Upgrade Protocol Checking}

A node which does not self-identify in PIP that it supports
the Firmware Upgrade Protocol should be considered to have passed these checks.
\pipsetFootnote

This test requires a vendor-provided current version of the 
appropriate firmware file for downloading into the node being checked.
Do not start this check without having that file available,
as a node is not required to recover from a firmware upgrade without
being successfully upgraded.

\subsection{Firmware Upgrade status}

This check follows the sequence documented in Section 5.5

At this time, only the datagram-based transfers are checked.

\begin{enumerate}
\item The checker sends a Memory Configuration datagram command “Freeze” with an 
    argument of Firmware Space 0xEF. 
    A Datagram Received reply may or many not be received in response.
\item The checker waits up to 20 seconds for a Node Initialization Complete message from the 
    node being checked.
\item The checker confirms that the PIP Firmware Upgrade active bit is set.
\item The checker uses Memory Write operations of 64 bytes to download the contents of the firmware file.
\item The checker resets the node being checked using a Memory Configuration 
    datagram command “Unfreeze” with an argument of Firmware Space 0xEF.    
    A Datagram Received reply may or many not be received in response.
\item The checker waits up to 20 seconds for a Node Initialization Complete message from the 
    node being checked.
\item The checker confirms that the PIP Firmware Upgrade active bit is reset.
\end{enumerate}

\end{document}  
