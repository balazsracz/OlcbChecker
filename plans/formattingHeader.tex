\documentclass[11pt]{article}
\usepackage{geometry}                % See geometry.pdf to learn the layout options. There are lots.
\geometry{letterpaper}                   % ... or a4paper or a5paper or ... 
%\geometry{landscape}                % Activate for for rotated page geometry
\usepackage[parfill]{parskip}    % Activate to begin paragraphs with an empty line rather than an indent
\usepackage{graphicx}
\usepackage{amssymb}
\usepackage{epstopdf}
\usepackage{hyperref}
\DeclareGraphicsRule{.tif}{png}{.png}{`convert #1 `dirname #1`/`basename #1 .tif`.png}

\usepackage{fancyhdr}
\fancyhf{}
\fancyfoot[R]{\thepage}
\renewcommand\headrulewidth{0pt}
\pagestyle{fancyplain}
% To put the name in the footer, do 
% \fancyfoot[L]{Foo}

% common header content elements
%\author{The OpenLCB Group}
%\date{}                                         % Activate to display a given date or no date



% to format XML tags in proper angle brackets <>
\newcommand*{\xml}[1]{\texttt{<#1>}}

% macro for inserting common introduction text
% takes hyperlink to a stadnard and the standard name as arguments
\newcommand{\introductionCaveats}[2]{

    \section{Introduction}

    This note documents the procedure for checking an OpenLCB implementation against the 
    \href{#1}
    {#2}.

    The checks are traceable to specific sections of the Standard.

    The checking assumes that the Device Being Checked (DBC) is being exercised by other
    nodes on the message network, 
    e.g. is responding to enquiries from other parts of the message network.
}

% macro for inserting the procedure section
\newcommand{\checkProcedure}[1]{
%    Select ``#1" in the program, 
%    then select each section below in turn.  Follow the prompts
%    for when to reset/restart the node and when to check 
%    outputs against the node documentation.
}

% macro deciding whether the pipset footnote should appear or not
\newcommand{\pipsetFootnote}{
%\footnote{Using the -p option or setting the checkpip default value False will skip this check.}
}