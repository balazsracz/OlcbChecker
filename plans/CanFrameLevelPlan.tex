\documentclass[11pt]{article}
\usepackage{geometry}                % See geometry.pdf to learn the layout options. There are lots.
\geometry{letterpaper}                   % ... or a4paper or a5paper or ... 
%\geometry{landscape}                % Activate for for rotated page geometry
\usepackage[parfill]{parskip}    % Activate to begin paragraphs with an empty line rather than an indent
\usepackage{graphicx}
\usepackage{amssymb}
\usepackage{epstopdf}
\usepackage{hyperref}
\DeclareGraphicsRule{.tif}{png}{.png}{`convert #1 `dirname #1`/`basename #1 .tif`.png}

\usepackage{fancyhdr}
\fancyhf{}
\fancyfoot[R]{\thepage}
\renewcommand\headrulewidth{0pt}
\pagestyle{fancyplain}
% To put the name in the footer, do 
% \fancyfoot[L]{Foo}

% common header content elements
%\author{The OpenLCB Group}
%\date{}                                         % Activate to display a given date or no date



% to format XML tags in proper angle brackets <>
\newcommand*{\xml}[1]{\texttt{<#1>}}

% macro for inserting common introduction text
% takes hyperlink to a stadnard and the standard name as arguments
\newcommand{\introductionCaveats}[2]{

    \section{Introduction}

    This note documents the procedure for checking an OpenLCB implementation against the 
    \href{#1}
    {#2}.

    The checks are traceable to specific sections of the Standard.

    The checking assumes that the Device Being Checked (DBC) is being exercised by other
    nodes on the message network, 
    e.g. is responding to enquiries from other parts of the message network.
}

% macro for inserting the procedure section
\newcommand{\checkProcedure}[1]{
%    Select ``#1" in the program, 
%    then select each section below in turn.  Follow the prompts
%    for when to reset/restart the node and when to check 
%    outputs against the node documentation.
}

% macro deciding whether the pipset footnote should appear or not
\newcommand{\pipsetFootnote}{
%\footnote{Using the -p option or setting the checkpip default value False will skip this check.}
}

\titleandheader{Checking the OpenLCB CAN Frame Level Protocols}

\begin{document}
\maketitle
\thispagestyle{firststyle}

\introductionCaveats
    {https://nbviewer.org/github/openlcb/documents/blob/master/standards/CanFrameTransferS.pdf}
    {CAN Frame Transfer Standard}.

\section{Frame Level Procedure}

\checkProcedure{Frame Layer Checking}

\subsection{Initialization}

This section's checks cover Frame Transfer Standard sections 4, 6.1 and section 6.2.1.

The checks assume that the node reserves a single alias at startup.

Follow the prompts when asked to reset or otherwise initialize the DBC.

The checker waits up to 30 seconds for the node to restart and 
go through a node reservation sequence.

\begin{enumerate}
\item All frames carry the same source alias
\item The sequence of four RID frames, a CID frame, and AMD frame are sent
\item The Node ID in the RID frames matches the Node ID in the AMD frame
\item That the Node ID matches that of the node being checked
\item Neither the alias\footnote{See section 6.3 of the Standard} 
nor the Node ID\footnote{See section 5.12 of the Unique Identifiers Standard.}
is zero.
\end{enumerate}


\subsection{AME Sequences}

This section's checks cover Frame Transfer Standard sections 4, 6.1 and section 6.2.3.

The checks assume that the node has previously reserved at least one alias
and is in the Permitted state.

The checker sends an AME frame with no NodeID and checks for:
\begin{enumerate}
\item An AMD frame in response
\item That carries the Node ID of the DBC
\end{enumerate}

The checker sends an AME frame with the Node ID of the DBC and checks the response for:
\begin{enumerate}
\item An AMD frame in response
\item That carries the Node ID of the DBC
\end{enumerate}

The checker sends an AME frame with a Node ID different from the Node ID of the DBC 
and checks for no response.


\subsection{Alias Conflict}

This section's checks cover Frame Transfer Standard sections 4, 6.1 and section 6.2.5.

The checks assume that the node has previously reserved at least one alias
and is in the Permitted state.

The checker sends an AME frame to acquire the DBC's current alias from the AMD
response.

The checker sends an CID frame with the DBC's alias and checks for
\begin{enumerate}
\item An RID frame in response
\item That carries the alias of the DBC.
\end{enumerate}

The checker sends an AMD frame with the DBC's alias and checks for
\begin{enumerate}
\item An AMR frame in response
\item That carries the source alias of the DBC.
\end{enumerate}

At this point, Frame Transfer Standard section 6.2.5 specifies that the node must stop
using that alias.  Many nodes will reserve a different one at this point.

If an alias reservation sequence does start, 
the first frame will be checked for a proper CID frame.  
In addition, the checker will check that the 
newly reserved alias is different from the original one.

Having reserved an alias, many nodes will put it into use with an AMD frame. 
If one is received, it will be checked to see if it contains the new alias
and the DBC's node ID.

Once traffic on the bus has paused, indicating the completion of 
any allocation(s), the checker will emit an AME frame.
\begin{enumerate}
\item If an AMD has been received, there should be exactly one
        reply with the DBC's node ID.  It should carry the new alias.
\item If an AMD has not been received, there should be no 
        replies with the DBC's node ID or with the new alias.
\end{enumerate}

Note: The Standards allow other initialization sequences, and this check
is not completely general.  If a DBC doesn't pass, the sequence should
be carefully examined in light of the CAN Frame Transfer Standard
to determine if it's just an issue with the check logic.
\subsection{Reserved Frame Bit}

This section's checks cover Frame Transfer Standard sections 4, 6.1 and section 6.2.3., 
specifically that the 0x1000\_0000 bit in the CAN header is properly
ignored.

The checker sends an AME frame with zero in the 0x1000\_0000 bit
and with no NodeID and checks for:
\begin{enumerate}
\item An AMD frame in response,
\item That carries the Node ID of the DBC,
\item With the 0x1000\_0000 bit set to one.
\end{enumerate}

\subsection{Capacity Check}

It is an unspoken assumption of the Standard that a node will receive and
process all CAN frames, even when the CAN bus is 100\% busy. (Though see Section 4
of the Technical Note for discussion on this.)

To confirm that, the checker:

\begin{enumerate}
    \item Sends 300 PCER messages with an Event ID of 
        00.00.00.00.00.00.00.01, followed by a Verify Node ID Number Addressed 
        message to the device being checked,
        followed by 300 PCER messages with an Event ID of 
        00.00.00.00.00.00.00.01. 
        All 601 messages are sent at the full CAN rate, with no inter-frame gaps.
    \item It then looks for whether a Verified Node ID Number message was transmitted in
        response to the Verify Node ID Number Addressed message. If not, the check fails.
    \item It then sends 300 Verify Node ID Number Addressed messages addressed 
        to another (not-present) node,
        followed by a Verify Node ID Number Global message, 
        then 300 Verify Node ID Number Addressed messages
        addressed to another (not-present) node.
        All 601 messages are sent at the full CAN rate, with no inter-frame gaps.
    \item It then looks for whether a Verified Node ID Number message was transmitted in
        response to the Verify Node ID Number Global message. If not, the check fails.
    \item It then sends 300 Identify Consumer messages with an Event ID of 
        00.00.00.00.00.00.00.01
        followed by a global Verify Nodes message, then 
        300 Identify Consumer messages with an Event ID of 
        00.00.00.00.00.00.00.01.
        All 601 messages are sent at the full CAN rate, with no inter-frame gaps.
    \item It then looks for whether a Verified Node ID Number message was transmitted in
        response to the global Verify Nodes message. If not, the check fails.
\end{enumerate}

\subsection{Compatibility with Standard CAN frames}

This checks the compatibility with standard CAN frames defined in section 4 of the standard.

The checker sends each valid standard CAN frame header, 2048 values, sequentially.
As it does this, it checks for a response from the node being checked.  Any response 
fails the check.

\end{document}  
