\documentclass[11pt]{article}
\usepackage{geometry}                % See geometry.pdf to learn the layout options. There are lots.
\geometry{letterpaper}                   % ... or a4paper or a5paper or ... 
%\geometry{landscape}                % Activate for for rotated page geometry
\usepackage[parfill]{parskip}    % Activate to begin paragraphs with an empty line rather than an indent
\usepackage{graphicx}
\usepackage{amssymb}
\usepackage{epstopdf}
\usepackage{hyperref}
\DeclareGraphicsRule{.tif}{png}{.png}{`convert #1 `dirname #1`/`basename #1 .tif`.png}

\usepackage{fancyhdr}
\fancyhf{}
\fancyfoot[R]{\thepage}
\renewcommand\headrulewidth{0pt}
\pagestyle{fancyplain}
% To put the name in the footer, do 
% \fancyfoot[L]{Foo}

% common header content elements
%\author{The OpenLCB Group}
%\date{}                                         % Activate to display a given date or no date



% to format XML tags in proper angle brackets <>
\newcommand*{\xml}[1]{\texttt{<#1>}}

% macro for inserting common introduction text
% takes hyperlink to a stadnard and the standard name as arguments
\newcommand{\introductionCaveats}[2]{

    \section{Introduction}

    This note documents the procedure for checking an OpenLCB implementation against the 
    \href{#1}
    {#2}.

    The checks are traceable to specific sections of the Standard.

    The checking assumes that the Device Being Checked (DBC) is being exercised by other
    nodes on the message network, 
    e.g. is responding to enquiries from other parts of the message network.
}

% macro for inserting the procedure section
\newcommand{\checkProcedure}[1]{
%    Select ``#1" in the program, 
%    then select each section below in turn.  Follow the prompts
%    for when to reset/restart the node and when to check 
%    outputs against the node documentation.
}

% macro deciding whether the pipset footnote should appear or not
\newcommand{\pipsetFootnote}{
%\footnote{Using the -p option or setting the checkpip default value False will skip this check.}
}

\title{Checking the OpenLCB Event Transport Protocol Standard}
\fancyfoot[L]{Checking the OpenLCB Event Transport Protocol Standard}

\begin{document}
\maketitle

\introductionCaveats
    {https://nbviewer.org/github/openlcb/documents/blob/master/standards/EventTransportS.pdf}
    {Event Transport Standard}.

\section{Event Transport Procedure}

\checkProcedure{Event Checking}

A node which does not self-identify in PIP that it supports
Event Transfer will be considered to have passed these checks.
\pipsetFootnote

A node which does self-identify in PIP that it supports 
Event Transfer is expected to consume or produce at least 
one event.  The checks are structured to check for that.

\textbf{Note:}  Proper handling of known events should be addressed.

\textbf{Note:}  This does not address the proper use of Unique IDs for Event IDs.

\subsection{Identify Events Addressed}

This section checks the addressed interaction in Standard section 6.2 and
the message formats in Standard section 4.3 through 4.8.

The check starts by sending an Identify Events message addressed to the DBC.
It then checks

\begin{enumerate}
\item That one or more Producer Identified, Producer Range Identified, 
        Consumer Identified and/or Consumer Range Identified messages are returned,
\item That those show the DBC node as their source.
\end{enumerate}

\subsection{Identify Events Global}

This section checks the unaddressed (global) interaction in Standard section 6.2 and
the message formats in Standard section 4.3 through 4.8.

The check starts by sending an Identify Events unaddressed (global) message.
It then checks

\begin{enumerate}
\item That one or more Producer Identified, Producer Range Identified, 
        Consumer Identified and/or Consumer Range Identified messages are returned,
\item That those show the DBC node as their source,
\item That these identify the same events produced and consumed as the 
        addressed form of the Identify Events message.
\end{enumerate}

\subsection{Identify Producer}

This section checks the interaction in Standard section 6.3, and
the message formats in Standard section 4.5 through 4.7.

The check proceeds by sending multiple Identify Producers messages for 
the zero or more individual event IDs returned by an Identify Events message 
addressed to the DBC. If there are none of these, this check passes. If there
are one or more, it then checks:

\begin{enumerate}
\item That exactly one reply is received for each Identify Producers message sent.
\item That those show the DBC node as their source,
\item That these identify the same event ID as the corresponding Identify message.
\end{enumerate}

\subsection{Identify Consumer}

This section checks the interaction in Standard section 6.4, and
the message formats in Standard section 4.2 through 4.4.

The check proceeds by sending multiple Identify Consumers messages for 
the zero or more individual event IDs returned by an Identify Events message 
addressed to the DBC. If there are none of these, this check passes. If there
are one or more, it then checks:

\begin{enumerate}
\item That exactly one reply is received for each Identify Consumers message sent.
\item That those show the DBC node as their source,
\item That these identify the same event ID as the corresponding Identify message.
\end{enumerate}

\subsection{Initial Advertisement}

Follow the prompts when asked to reset or otherwise initialize the DBC.

This section's checks the interaction in the preamble to Standard section 6, and
the messages in Standard sections 4.1, 4.3, 4.4, 4.6 and 4.7.

\textbf{Note:}  There's no requirement that the Identify Producer messages
be sent immediately, only that they be sent before the events are produced.
Nodes typically send them immediately, and that's what this checks.

This check starts by restarting the node, which causes a transition to Initialized
state.  That is then followed by the node identifying events that it will 
produce and consume by appropriate messages. This then checks:

\begin{enumerate}
\item That the Producer Identified, Producer Identified Range, Consumer Identified 
    and Consumer Identified Range messages produced at node startup are the same
    as the ones emitted in response to an addressed Identify Events,
\item That those messages show the DBC as their source.
\item That no Producer Consumer Event Report messages are sent before the 
    corresponding producer has been identified.
\item That any PCER messages that are sent show the DBC as their source.
\end{enumerate}

\end{document}  
