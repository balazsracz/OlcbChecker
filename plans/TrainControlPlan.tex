\documentclass[11pt]{article}
\usepackage{geometry}                % See geometry.pdf to learn the layout options. There are lots.
\geometry{letterpaper}                   % ... or a4paper or a5paper or ... 
%\geometry{landscape}                % Activate for for rotated page geometry
\usepackage[parfill]{parskip}    % Activate to begin paragraphs with an empty line rather than an indent
\usepackage{graphicx}
\usepackage{amssymb}
\usepackage{epstopdf}
\usepackage{hyperref}
\DeclareGraphicsRule{.tif}{png}{.png}{`convert #1 `dirname #1`/`basename #1 .tif`.png}

\usepackage{fancyhdr}
\fancyhf{}
\fancyfoot[R]{\thepage}
\renewcommand\headrulewidth{0pt}
\pagestyle{fancyplain}
% To put the name in the footer, do 
% \fancyfoot[L]{Foo}

% common header content elements
%\author{The OpenLCB Group}
%\date{}                                         % Activate to display a given date or no date



% to format XML tags in proper angle brackets <>
\newcommand*{\xml}[1]{\texttt{<#1>}}

% macro for inserting common introduction text
% takes hyperlink to a stadnard and the standard name as arguments
\newcommand{\introductionCaveats}[2]{

    \section{Introduction}

    This note documents the procedure for checking an OpenLCB implementation against the 
    \href{#1}
    {#2}.

    The checks are traceable to specific sections of the Standard.

    The checking assumes that the Device Being Checked (DBC) is being exercised by other
    nodes on the message network, 
    e.g. is responding to enquiries from other parts of the message network.
}

% macro for inserting the procedure section
\newcommand{\checkProcedure}[1]{
%    Select ``#1" in the program, 
%    then select each section below in turn.  Follow the prompts
%    for when to reset/restart the node and when to check 
%    outputs against the node documentation.
}

% macro deciding whether the pipset footnote should appear or not
\newcommand{\pipsetFootnote}{
%\footnote{Using the -p option or setting the checkpip default value False will skip this check.}
}

\title{Checking the OpenLCB Train Control Protocol Standard}
\fancyfoot[L]{Checking the OpenLCB Train Control Protocol Standard}

\begin{document}
\maketitle


\introductionCaveats
    {https://nbviewer.org/github/openlcb/documents/blob/master/standards/TrainControlS.pdf}
    {Train Control Protocol Standard}

\section{Train Control Protocol Procedure}

\checkProcedure{Train Control Protocol Checking}

This procedure checks the operation of a Train Node.  

A node which does not self-identify in PIP that it supports
the Train Control Protocol should be considered to have passed these checks.
\pipsetFootnote

Note that this commands the speed and functions of a locomotive node.
Although the commanded speed is quite low and only for a short time, the
checks should be run in a way that the associated physical locomotive
does not run away.

\subsection{Defined Event ID checking}

This section checks that the node supports the isTrain event
defined in section 4.1 and 6.4 of the Train Control Protocol Standard.

It does this by issuing an Identify Events to the node, and then
checking for a Producer Identified reply carrying the isTrain Event ID.

\subsection{Check set and query speeds}

This checks the ability to set and query the speed and direction of the train.
Note that it checks that forward/reverse is independent of the speed setting, particularly
at zero.

\begin{enumerate}

    \item Set the speed and direction to 0.75 reverse.

    \item Query the speed and direction. Check for 0.75 reverse.

    \item Set the speed and direction to 0 reverse.

    \item Query the speed and direction. Check for 0 reverse.

    \item Set the speed and direction to 0.75 forward.

    \item Query the speed and direction. Check for 0.75 forward.

    \item Set the speed and direction to 0 forward.

    \item Query the speed and direction. Check for 0 forward.

\end{enumerate}

In any case, even if the earlier checks failed, end by setting the speed 
and direction to 0 forward.

\subsection{Check set and query of functions}

This section assumes that F0 is available on the train.
The FDI is not used to confirm that.

\begin{enumerate}

    \item Set F0 to on.

    \item Query F0 and check for an "on" response.

    \item Set F0 to off.

    \item Query F0 and check for an "off" response.

\end{enumerate}

In any case, even if the earlier checks failed, end by setting F0 to "off".

\subsection{Check Emergency Stop}

This checks the behavior defined in section 6.2 of the Standard.
It does not check that the train stops.

\begin{enumerate}

    \item Set the speed and direction to 0.75 reverse.

    \item Query the speed and direction. Check for 0.75 reverse.

    \item Send an emergency stop to the train.

    \item Query the speed and direction. Check for 0 reverse.

    \item Set the speed and direction to 0.75 forward.

    \item Query the speed and direction. Check for 0.75 forward.

    \item Set the speed and direction to 0 forward.

\end{enumerate}

In any case, even if the earlier checks failed, end by setting the speed 
and direction to 0 forward.

\subsection{Check Global Emergency Stop}

This checks the behavior defined in section 6.2 of the Standard.
It does not check that the train stops.

\begin{enumerate}

    \item Set the speed and direction to 0.75 reverse.

    \item Query the speed and direction. Check for 0.75 reverse.

    \item Produce an Emergency Stop All event.

    \item Query the speed and direction. Check for 0.75 reverse.

    \item Set the speed and direction to 0.75 forward.

    \item Query the speed and direction. Check for 0.75 forward.

    \item Produce a Clean Emergency Stop event.
    
    \item Set the speed and direction to 0 forward.

\end{enumerate}

In any case, even if the earlier checks failed, end by setting the speed 
and direction to 0 forward.

\subsection{Check Global Emergency Off}

This checks the behavior defined in section 6.2 of the Standard.
It does not check that the train stops and that the train's outputs power off.

\begin{enumerate}

    \item Set the speed and direction to 0.75 reverse.

    \item Query the speed and direction. Check for 0.75 reverse.

    \item Produce an Emergency Off All event.

    \item Query the speed and direction. Check for 0.75 reverse.

    \item Set the speed and direction to 0.75 forward.

    \item Query the speed and direction. Check for 0.75 forward.

    \item Produce a Clean Emergency Off event.
    
    \item Set the speed and direction to 0 forward.

\end{enumerate}

In any case, even if the earlier checks failed, end by setting the speed 
and direction to 0 forward.

\subsection{Check memory spaces}

This section checks the existence and properties of the memory spaces defined in section 7
of the Train Control Protocol Standard.

This does not check the information defined by the 
Function Definition Information Standard.

\begin{enumerate}

\item Use a Get Address Space Information Command datagram from the Memory Configuration protocol
to enquire about the status of space 0xF9.

\item Check that the response indicates the memory space exists.

\item Use a Get Address Space Information Command datagram from the Memory Configuration protocol
to enquire about the status of space 0xFA.

\item Check that the response indicates the memory space exists.

\end{enumerate}
    
\subsection{Checking function to/from memory space connection}

This section assumes that F0 is available on the train.
The FDI is not used to confirm that.

\begin{enumerate}

    \item Set function 0 "on"

    \item Check byte 0 in the 0xF9 memory space for a non-zero value.

    \item Write byte 0 in the 0xF9 memory space to zero.

    \item Check that function 0 is "off".

\end{enumerate}


\subsection{Check Controller Configuration command and response}

\begin{enumerate}

    \item Send a Train Control Assign Controller command with the checker's Node ID.
    \item Check for a Train Control Assign Controller reply with OK in the flags.

    \item Send a Train Control Query Controller command
    \item Check for a Train Control Query Controller response with the checker's Node ID
        in the Active Controller field.

    \item Send a Train Control Release Controller command
    \item Check for a Train Control Query Controller response with a zero Node ID
        in the Active Controller field.
    
    \item Send a Train Control Assign Controller command with the checker's Node ID.
    \item Check for a Train Control Assign Controller reply with OK in the flags.

\end{enumerate}

Note:  There needs to be additional checking of the Controller Changing Notify 
    added here.

In any case, including if some check fails, end by sending a Train Control Release
    Controller command.

\subsection{Check Listener Configuration command and response}

This section is still to be written.

\subsection{Check Train Control Management command and response}

This assumes that the node being checked has not been reserved by another node.

\begin{enumerate}

    \item Send a Train Control Management Reserve command
    \item Check for a Train Control Management Reserve response with OK in the flags.

    \item Send a Train Control Management Release command.  No response is expected.

    \item Send a Train Control Management Reserve command
    \item Check for a Train Control Management Reserve response with OK in the flags.

    \item Send a Train Control Management Reserve command
    \item Check for a Train Control Management Reserve response with fail in the flags.
    
    \item Send a Train Control Management Release command.  No response is expected.

\end{enumerate}

\end{document}  
