\documentclass[11pt]{article}
\usepackage{geometry}                % See geometry.pdf to learn the layout options. There are lots.
\geometry{letterpaper}                   % ... or a4paper or a5paper or ... 
%\geometry{landscape}                % Activate for for rotated page geometry
\usepackage[parfill]{parskip}    % Activate to begin paragraphs with an empty line rather than an indent
\usepackage{graphicx}
\usepackage{amssymb}
\usepackage{epstopdf}
\usepackage{hyperref}
\DeclareGraphicsRule{.tif}{png}{.png}{`convert #1 `dirname #1`/`basename #1 .tif`.png}

\usepackage{fancyhdr}
\fancyhf{}
\fancyfoot[R]{\thepage}
\renewcommand\headrulewidth{0pt}
\pagestyle{fancyplain}
% To put the name in the footer, do 
% \fancyfoot[L]{Foo}

% common header content elements
%\author{The OpenLCB Group}
%\date{}                                         % Activate to display a given date or no date



% to format XML tags in proper angle brackets <>
\newcommand*{\xml}[1]{\texttt{<#1>}}

% macro for inserting common introduction text
% takes hyperlink to a stadnard and the standard name as arguments
\newcommand{\introductionCaveats}[2]{

    \section{Introduction}

    This note documents the procedure for checking an OpenLCB implementation against the 
    \href{#1}
    {#2}.

    The checks are traceable to specific sections of the Standard.

    The checking assumes that the Device Being Checked (DBC) is being exercised by other
    nodes on the message network, 
    e.g. is responding to enquiries from other parts of the message network.
}

% macro for inserting the procedure section
\newcommand{\checkProcedure}[1]{
%    Select ``#1" in the program, 
%    then select each section below in turn.  Follow the prompts
%    for when to reset/restart the node and when to check 
%    outputs against the node documentation.
}

% macro deciding whether the pipset footnote should appear or not
\newcommand{\pipsetFootnote}{
%\footnote{Using the -p option or setting the checkpip default value False will skip this check.}
}

\titleandheader{Checking the OpenLCB Function Description Information Standard}

\begin{document}
\maketitle
\thispagestyle{firststyle}

\introductionCaveats
    {https://nbviewer.org/github/openlcb/documents/blob/master/standards/FunctionDescriptionInformationS.pdf}
    {Function Definition Information Standard}

\section{Function Description Information Procedure}

\checkProcedure{Function Description Information Checking}

This plan assumes that the Datagram Transport Protocol and the Memory Configuration Protocol 
have been separately checked. It uses those, but does not do any detailed checking of them.

\subsection{FDI Memory Present checking}

This section checks that the FDI memory defined in the Function 
Definition Information Standard is present.

The Get Address Space Information Command from the Memory Configuration Protocol
is used to validate that space 0xFA is present and read-only.

\subsection{Validation checking}

This section checks the content of the FDI against the XML Schema 
defined in Section 5 of the Standard to make sure that it is valid XML. 

This check reads the information from the 0xFA memory space
until it encounters a null character.
\begin{enumerate}
\item If the space is not available, or there are no bytes of content in it, the check fails.
\item If the read sequence terminates early, without ending in a null character, the check fails.
\end{enumerate}
 
The check then compares the first two lines of the content against the 
definition in section 5. Any XML Schema version up to and including the 
present one is permitted.

The check then validates the content against the 1.0 FDI XML Schema
which is stored in a local ”fdi.xsd” file.
(This should pull the CDI from the web)
(This has to handle various versions of the schema)

The check then compares the length of the 0xFA memory space 
as specified by the Memory Configuration Protocol Standard sections 4.15 and 4.16
vs. the actual length of the retrieved FDI.
If they do not agree, the check fails.

\end{document}  
