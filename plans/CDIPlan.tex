\documentclass[11pt]{article}
\usepackage{geometry}                % See geometry.pdf to learn the layout options. There are lots.
\geometry{letterpaper}                   % ... or a4paper or a5paper or ... 
%\geometry{landscape}                % Activate for for rotated page geometry
\usepackage[parfill]{parskip}    % Activate to begin paragraphs with an empty line rather than an indent
\usepackage{graphicx}
\usepackage{amssymb}
\usepackage{epstopdf}
\usepackage{hyperref}
\DeclareGraphicsRule{.tif}{png}{.png}{`convert #1 `dirname #1`/`basename #1 .tif`.png}

\usepackage{fancyhdr}
\fancyhf{}
\fancyfoot[R]{\thepage}
\renewcommand\headrulewidth{0pt}
\pagestyle{fancyplain}
% To put the name in the footer, do 
% \fancyfoot[L]{Foo}

% common header content elements
%\author{The OpenLCB Group}
%\date{}                                         % Activate to display a given date or no date



% to format XML tags in proper angle brackets <>
\newcommand*{\xml}[1]{\texttt{<#1>}}

% macro for inserting common introduction text
% takes hyperlink to a stadnard and the standard name as arguments
\newcommand{\introductionCaveats}[2]{

    \section{Introduction}

    This note documents the procedure for checking an OpenLCB implementation against the 
    \href{#1}
    {#2}.

    The checks are traceable to specific sections of the Standard.

    The checking assumes that the Device Being Checked (DBC) is being exercised by other
    nodes on the message network, 
    e.g. is responding to enquiries from other parts of the message network.
}

% macro for inserting the procedure section
\newcommand{\checkProcedure}[1]{
%    Select ``#1" in the program, 
%    then select each section below in turn.  Follow the prompts
%    for when to reset/restart the node and when to check 
%    outputs against the node documentation.
}

% macro deciding whether the pipset footnote should appear or not
\newcommand{\pipsetFootnote}{
%\footnote{Using the -p option or setting the checkpip default value False will skip this check.}
}

\titleandheader{Checking the OpenLCB Configuration Description Information Standard}

\begin{document}
\maketitle
\thispagestyle{firststyle}

\introductionCaveats
    {https://nbviewer.org/github/openlcb/documents/blob/master/standards/ConfigurationDescriptionInformationS.pdf}
    {Configuration Description Information (CDI) Standard}.

\section{Configuration Description Information Procedure}

\checkProcedure{CDI Checking}

This plan assumes that the Datagram Transport Protocol and the Memory Configuration Protocol 
have been separately checked. It uses those, but does not do any detailed checking of them.
\footnote{Those checks don't check for write capability, because at that level
    there's no reliable way to know that it's possible. 
    Once ACDI is established below, we could use writes to and reads from the
    251 space to check the ability to write.
    That requires further discussion.
}

\subsection{CDI Memory Present checking}

This section checks that the CDI memory defined in section 4.2 of the Memory
Configuration Protocol Standard is present.

A node which does not self-identify in PIP that it supports
Configuration Description Information passes this check.

The Get Address Space Information Command from the Memory Configuration Protocol
is used to validate that space 0xFF is present and read-only.

\subsection{Validation checking}

This section checks the content of the CDI against the XML Schema 
defined in Section 5 of the Standard to make sure that it is valid XML. 

A node which does not self-identify in PIP that it supports
Configuration Description Information passes this check.

A node which has the CDI bit set in PIP, but does
has the Memory Configuration bit reset in PIP, fails this check.
\pipsetFootnote

This check reads the information from the 0xFF memory space
until it encounters a null character.
\begin{enumerate}
\item If the space is not available, or there are no bytes of content in it, the check fails.
\item If the read sequence terminates early, without ending in a null character, the check fails.
\end{enumerate}
 
The check then validates the content against the 1.3 CDI XML Schema
which is stored in a local ”cdi.xsd” file.

The check then compares the length of the 0xFF memory space 
as specified by the Memory Configuration Protocol Standard sections 4.15 and 4.16
vs. the actual length of the retrieved CDI.
If they do not agree, the check fails.

\subsection{ACDI checking}

This checks against the statements in section 5.1.2 of the Standard.

Depending on the bits set in the PIP information, there are several 
possible configurations to check. Each of these is checked in turn, if relevant.

\begin{enumerate}

\item ACDI bit set and Memory Configuration bit not set

This configuration fails the check.  PIP ACDI set implies that certain information 
is available in the configuration memory.

\item ACDI bit set and Memory Configuration bit set

The 251 (0xFB) and 252 (0xFC) memory spaces are checked for
being present in the Memory Configuration Protocol.

The two version numbers in the 251 (0xFB) and 252 (0xFC) memory spaces
are read and checked against their required values.

The six ACDI-defined strings are read from their respective memory spaces
to make sure that can be done without error.

\item SNIP bit, ACDI bit and Memory Configuration bit all set

The SNIP data is acquired.
 
The six SNIP strings are checked for equality, up to the first zero byte, to the
strings acquired from memory in the step above.

\item Memory Configuration bit and CDI bit set

The check reads the first 820 bytes of the CDI information in space 255 (0xFF) to 
determine if an \xml{acdi} element is present
\footnote{The proper placement of the \xml{acdi} element was checked in the Validation 
checking step above}.  

If the \xml{acdi} element is present and the ACDI bit is not set in PIP, or
if the \xml{acdi} element is not present and the ACDI bit is set in PIP,
the check fails.

\end{enumerate}

\subsection{Identification Element checking}

This checks against the statements in section 5.1.1 of the Standard.

A node which does not self-identify in PIP that it supports
Configuration Description Information passes this check.
\pipsetFootnote

The check reads the first 480 bytes of the CDI information in space 255 (0xFF) to 
determine if an \xml{identification} element is present
\footnote{The proper placement and contents of the \xml{identification} element
was checked in the Validation checking step above}.  

If the \xml{identification} element is not present, this check pases.

If the SNIP bit is set in PIP, the contents of the \xml{identification}
sub-elements are checked against the SNIP contents
up to the first zero byte.

If the \xml{acdi} element is present, the contents of the 
\xml{identification} sub-elements are checked against the ACDI-defined
contents of the 251 (0xFB) memory space up to the first zero byte.

\end{document}  
