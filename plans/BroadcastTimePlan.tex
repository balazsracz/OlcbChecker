\documentclass[11pt]{article}
\usepackage{geometry}                % See geometry.pdf to learn the layout options. There are lots.
\geometry{letterpaper}                   % ... or a4paper or a5paper or ... 
%\geometry{landscape}                % Activate for for rotated page geometry
\usepackage[parfill]{parskip}    % Activate to begin paragraphs with an empty line rather than an indent
\usepackage{graphicx}
\usepackage{amssymb}
\usepackage{epstopdf}
\usepackage{hyperref}
\DeclareGraphicsRule{.tif}{png}{.png}{`convert #1 `dirname #1`/`basename #1 .tif`.png}

\usepackage{fancyhdr}
\fancyhf{}
\fancyfoot[R]{\thepage}
\renewcommand\headrulewidth{0pt}
\pagestyle{fancyplain}
% To put the name in the footer, do 
% \fancyfoot[L]{Foo}

% common header content elements
%\author{The OpenLCB Group}
%\date{}                                         % Activate to display a given date or no date



% to format XML tags in proper angle brackets <>
\newcommand*{\xml}[1]{\texttt{<#1>}}

% macro for inserting common introduction text
% takes hyperlink to a stadnard and the standard name as arguments
\newcommand{\introductionCaveats}[2]{

    \section{Introduction}

    This note documents the procedure for checking an OpenLCB implementation against the 
    \href{#1}
    {#2}.

    The checks are traceable to specific sections of the Standard.

    The checking assumes that the Device Being Checked (DBC) is being exercised by other
    nodes on the message network, 
    e.g. is responding to enquiries from other parts of the message network.
}

% macro for inserting the procedure section
\newcommand{\checkProcedure}[1]{
%    Select ``#1" in the program, 
%    then select each section below in turn.  Follow the prompts
%    for when to reset/restart the node and when to check 
%    outputs against the node documentation.
}

% macro deciding whether the pipset footnote should appear or not
\newcommand{\pipsetFootnote}{
%\footnote{Using the -p option or setting the checkpip default value False will skip this check.}
}

\title{Checking the OpenLCB Broadcast Time Protocol Standard}

\begin{document}
\maketitle


\introductionCaveats
    {https://nbviewer.org/github/openlcb/documents/blob/master/standards/BroadcastTimeS.pdf}
    {Broadcast Time Protocol Standard}

\section{Broadcast Time Protocol Producer checking}

The Broadcast Time Protocol specifies the behavior of both producers and consumers of 
broadcast time information.  
The producer behavior is checked in this section.

These checks should be done for each defined clock separately. 
For example, they can be run on the 01.01.00.00.01.00 Default Fast Clock
followed by running them for the 01.01.00.00.01.01 Default Real-time Clock.
Consult the device's documentation to see which clock IDs are defined.

\checkProcedure{Broadcast Time Protocol Checking}

\subsection{Clock Query checking}

This checks that the Clock Query defined in Section 6.4 of the Standard results in the
Clock Synchronization sequence defined in section 6.3.

The clock query is sent. 
Then the specific reports in items 1 through 6 are checked for, in order.

Note that the values of the time, date, rate, state are not checked in this step.

\subsection{Clock Set checking}

This checks that the Clock Set operations defined in Section 6.5 of the Standard results 
in new clock settings followed by the
Clock Synchronization sequence defined in section 6.3.

This then sets the clock to 01:23 AM on July 22, 2023 with the rate set to 4X and the clock stopped.

The check then waits up to 4 seconds for a Clock Synchronization sequence, which it checks for
the specific time, date, rate and state that was previously set.

\subsection{Requested Time checking}

This checks that the Report Time Event indicated in bullet 2 of section 6.2 is 
produced when requested.

First this check sends a Consumer Identified message for the 01:24AM  July 22, 2023 event.

Then it sets the clock to 01:23 AM on July 22, 2023 with the rate set to 5X and the clock running.

It then waits up to 30 seconds for the production of the 01;24AM Report Time Event.
If the event is not received, or it contains the wrong contents, the check fails.

\subsection{Startup Sequence checking}

This checks the startup sequence described in section 6.1 of the Standard.

The node being checked is manually restarted.

The resulting message traffic is then checked for the required elements:
\begin{enumerate}
\item The Producer Range Identified and Consumer Range Identified messages, in either order.
\item A Clock Synchronization sequence as defined in section 6.3.
\end{enumerate}

Note that the existence of the Clock Synchronization messages are checked, but not 
their contents.  The Clock is allowed to start up with any time value.

\section{Broadcast Time Protocol Consumer checking}

The Broadcast Time Protocol specifies the behavior of both producers and consumers of 
broadcast time information.  
The consumer behavior is checked in this section.

These checks should be done for each defined clock separately. 
For example, they can be run on the 01.01.00.00.01.00 Default Fast Clock
followed by running them for the 01.01.00.00.01.01 Default Real-time Clock.
Consult the device's documentation to see which clock IDs are defined.

\subsection{Synchronization Checking}

This checks the response of the clock to the Clock Synchronization sequence defined
in section 6.3.

The check sends the synchronization sequence for 
    01:23 AM on July 22, 2023 with the rate set to 4X and the clock stopped.
    
The user is then prompted to check the consumer node's display, if any, to see if the
same time, date, rate and/or state are being displayed.



\end{document}  
